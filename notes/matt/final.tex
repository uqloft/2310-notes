\documentclass[10pt, a4paper]{article}

\def\name{\textbf{2310}}
\def\auth{Matthew Low}

\usepackage[T1]{fontenc}
\usepackage[shortlabels]{enumitem}
\usepackage[utf8]{inputenc}
\usepackage[letterspace=-50]{microtype}
\usepackage{amsmath}
\usepackage{amssymb}
\usepackage{amsthm}
\usepackage{color}
\usepackage{float}
\usepackage{framed}
\usepackage{geometry}
\usepackage{listings}
\usepackage{multicol}
\usepackage{sectsty}
\usepackage{xcolor}
\usepackage{titlesec}

\lstset{language=C, xleftmargin=0em, numbers=none, basicstyle=\linespread{0.9}\ttfamily, breaklines=true,
postbreak=\mbox{\textcolor{gray}{$\hookrightarrow$}\space}}
% \usepackage{newpxtext, newpxmath}
% \usepackage[cm]{sfmath}
\usepackage{mathspec}
\setmainfont{Helvetica Neue}
\setmonofont{SF Mono}
\setmathrm{Helvetica Neue}
\setlist{leftmargin=4mm, nosep}
\allsectionsfont{\color{blue}}
% \renewcommand{\familydefault}{\sfdefault}

\pagestyle{empty}
\geometry{margin=0.5cm}
\setlength{\parindent}{0cm}
\setlength{\columnsep}{0.2cm}
\linespread{1}

\begin{document}

\huge

    \addfontfeature{LetterSpace=-5}
    \textbf{\color{blue}CSSE2310} Computer Systems Principles and Programming
    
    \addfontfeature{LetterSpace=0}
    \color{blue}
    \rule{\textwidth}{1.5em}
    \color{black}
    \tiny

\begin{multicols}{4}
    \subsection*{Question Tips}
    \begin{itemize}
        \item Unused addr: $2^{\text{host bits}}- 2 - \text{number of addresses}$.
        \item Thread diagrams are always valid.
        \item socket, bind, listen, accept
        \item socket, connect
        \item LOOK AT PRE AND POST INCREMENT!!!!
        \item TRANSLATION LOOKASIDE BUFFER
        \item TCP: send message back \textbf{always}.
        \item TLB: Memory cache for pages and page to frame references.
        \item One level: if in the TLB 0, otherwise 1 page read
        \item Two level: if in the TLB 0, otherwise, 2 pages read
        \item Number of memory access: 1 access if in the TLB, 3 if two-level,2 if one-level
    \end{itemize}
    

    \subsection*{Unix}
    Examples (courtesy of Ainsley Nand on Attic)
    \begin{itemize}
    \item Select the first and third column delimited by ‘ ‘
    and sort by the first

    cat file.txt | cut -d’ ‘ -c1,3 | sort -k1

    \item Get only the 12th line of a given file

    cat file.txt | head -12 | tail -1

    \item Get each line with cat and dog in a given file

    cat file.txt | grep “cat” | grep “dog”

    \item Find each line which contains “CSSE1001” and does
    not contain the word “boring”

    cat file.txt | grep “CSSE1001” | grep -v “boring”

    \item Find the number of times “rowing” occurs
    in a given file

    cat ainsley.txt | grep -o “rowing” | wc -l

    \item For all files f1,f2,f3 show all lines containing
    “song”, “river” and “terrible”

    cat f1 f2 f3 | grep song | grep river
    | grep terrible

    \item For all files g1,g2,g3 show all lines not containing
    “song”, “river” and “awful”

    cat g1 g2 g3 | grep -v song | grep -v river 
    | grep -v awful
    OR	grep -ve grep -ve song -ve awful f1 f2 f3

    \item Find all lines in file1 containing the word
    “dinosaur” and store in file called london

    cat file1 | grep “dinosaur” > london  
    OR	grep dinosaur file1 > london

    \item Show all lines in file1 starting with W

    grep \^{}W file1

    \item Show all lines in file2 ending with S

    grep S\$ file2

    \item Modify path

    export PATH = \$PATH:newpath

    \item Show second last line of file

    cat file | tail -n 2 | head -n 1

    \item Show fifth line of file

    cat file | head -n 5 | tail -n 1

    \item Show the third line and later
    (hide the first 2 lines)

    cat file | tail -n +3

    \item Show all but the last 10 lines
    (hide the last 10 lines)

    cat file | head -n -10      
    \end{itemize}
      
    Unix commands:

    \begin{itemize}
        \item grep  print lines matching pattern
        \begin{itemize}
        \item  -v  invert-match: displays lines not containing the string 
        \item  -o  only show part of line that matches pattern 
        \item  -i  case insensitive 
        \item  -R  search all directories 
        \item  \$  fish\$, for example, looks for the word fish as the last word of a line 
        \item  \^{}  fish\^{}, for example, looks for the word fish as the first word of a line 
        \item  .  any character in regex except newlines 
        \item  *  0 or more of previous expression in regex 
        \end{itemize}
        \item gcc  create executable
        \begin{itemize}
        \item  -o  cretaes an executable file from a c file, the order needs to be: \texttt{gcc -o <executable-name> <name-of-file>} 
        \item  -c  compiles a c file and creates an .o file of the same name 
        \item  -g  include debugging symbols \end{itemize} 
        \item ls  list directory contents
        \begin{itemize}
        \item  -l  use a long listing format 
        \item  -a  shows all hidden directories, do not ignore entries starting with . 
        \item  -d  list directories themselves not their content 
        \item  -i  print the index number of each file 
        \end{itemize}
        \item ps  process status
        \begin{itemize}
        \item  -e  show all processes (-> to see every process on the system using standard syntax) 
        \item  -u  user's processes, if wishing to specify a user, specify -u  
        \item  -f  do full format listing, adds additional columns 
        \end{itemize}
        \item sort  write sorted concatenation to stdout
        \begin{itemize}
        \item  -r  reverse result of comparisons 
        \item  -k  sort via key 
        \end{itemize}
        \item uniq  report or omit repeated lines
        \begin{itemize}
        \item  -c  count number of occurrences
        \end{itemize}
        \item cat  concatenate files to print on stdout
        \item head  output first part of file to stdout (def. 10)
        \begin{itemize}
        \item  -n  output first N number of lines (with leading - print all but last N) 
        \item  -c  print first N bytes of each file (leading - as above)
        \item  -q  quiet, never print headers giving file name
        \end{itemize}
        \item tail  output last part of file to stdout (def. 10)
        \begin{itemize}
        \item  -n  output last N number of lines 
        \end{itemize}
        \item cut  remove sections from each line of flines
        \begin{itemize}
        \item  -f  select only these fields; print any line that contains no delimiter character, unless -s is specified. Can specify multiple fields with a comma (e.g. -f1,3,4 ). Can specify ranges with dashes. N- Nth onwards, -M up to M, N-M n to m
        \item  -s  do not print lines not containing delimiters 
        \item  -d  specify delimiter (e.g. -d':')
        \end{itemize}
        \item wc  print newline, word and byte counts for each file
        \begin{itemize}
        \item  -l  number of newlines 
        \item  -c  byte count 
        \item  -m  character count 
        \end{itemize}
        \item diff  compare files line by line
        \begin{itemize}
        \item  -q  report only when files differ 
        \item  -s  report identical files 
        \end{itemize}
        \item svn  subversion
        \begin{itemize}
        \item commit  send changes from working copy to repository
        \item add put files in directory under version control. Added to repository in next commit
        \item remove remove files and directories from VC. Scheduled for deletion upon next commit and removed from working copy.
        \item move move and/or name something in working copy or repository
        \item update bring changes from repository into working copy
        \item info display information about a local or remote item
        \item log show log messages for a set revision(s) and/or path(s)
        \item status print the status of working copy and files and directories
        \item diff display differences between two revisions or paths
        \end{itemize}
        \item chmod  change file modes (permissions)
        \begin{itemize}
            \item u/g/o/a user/group/others/all
            \item +/- add remove
            \item r/w/x read/write/execute
            \item -c only report when a change is made
            \item -R change files and directories recurisvely
            \item -v output a diagnostic for eveyr file processed
            \item -f suppress most error messages
        \end{itemize}
        \item rm  remove
        \begin{itemize}
            \item -r remove directories and their contents recursively, i.e. delete everything in the specified subdirectories
            \item -f force -> ignore nonexistent files and arguments, never prompt
            \item -d remove empty directories
            \item -v explain what is being done (verbose)
            \item * remove all
        \end{itemize}
        \item mkdir  make directories
        \begin{itemize}
            \item -m set file mode like chmod
            \item -p no error if existing parent, make parent directories as needed
        \end{itemize}
        \item rmdir  remove empty directories
        \begin{itemize}
            \item -p remove directory and its ancestors e.g. `rmdir -p a/b/c' becomes `rmdir a/b/c a/b a'
        \end{itemize}
        \item cp  copy files
        \begin{itemize}
            \item -r copy directories recursively
        \end{itemize}
        \item scp  secure copy
        \begin{itemize}
            \item -c selects the cipher to use for encrypting data, this option is directly passed to ssh(1)
        \end{itemize}
        \item mv  move files / rename
        \begin{itemize}
            \item -f force; do not prompt before overwriting
            \item -i interactive; prompt before overwrite
        \end{itemize}
        \item vim  programmers text editor
        \item pico  simple text editor
        \item less  allows backward and forward movement
        \item ln  makes links between files (source is first, destination is second) 
        \begin{itemize}
            \item -s for symbolic link
            \item -p for physical/hard link
        \end{itemize}
    \end{itemize}
    
    \subsection*{Networks}
    Five-layer TCP/IP stack:
    \begin{enumerate}
        \item \textbf{Physical layer}
            Medium through which signals travel through.
        \item \textbf{(Data-)link layer} where peers can communicate directly via messages (MAC addresses)
        \item \textbf{Network layer} exchange messages with any other host on the ``internet'', uses IP protocol, IPv4 addresses are 32bit dotted quads, sends messages via the link layer.
        \item \textbf{Transport layer} UDP user datagram protocol (datagrams)/TCP transmission control protocol (segments), sends messages using the network layer, addresses are ports (16 bit int, restricted below 1024)
        \item \textbf{Application layer} ``Everything else'' / web, ssh, games, SMB. addresses include URL/URI but can be anything.
    \end{enumerate}
    We have four main addresses:
    \begin{itemize}
        \item Application specific addresses
        \item Port (differentiate between processes)
        \item IP (which computer is this process on)
        \item MAC (which device is this direct message to)
    \end{itemize}
    IP addresses and MAC addresses technically don't identify devices. The layer isolation is not completely enforced.

    \textbf{What is the transport layer for?} The network layer deals with packets. We'd like streams of bytes and reliable delivery. TCP is good for this.
    \begin{itemize}
        \item Establish connection
        \item Making a connection requires messages to travel there and back
        \item Connections are bi-directional
    \end{itemize}
    UDP deals with discrete messages and no guarantees of delivery or acknowledgment. So streaming video, games, congested networks (many small operations) UDP is appropriate for.

    \[\mathrm{Bandwidth} \neq \mathrm{Latency}\]
    Where do ports come from?
    \begin{itemize}
        \item Ports: you choose
        \item IP: internal (admin chooses) or external (DHCP)
        \item MAC: default address, can be changed
    \end{itemize}
    IPv4 has 32 bit addresses. IPv6 is 128bit. We use IPv4.
    \textbf{Server} a process which waits for requests from  clients. \textbf{Client} a process which submits requests to a server.
    
    TCP: there is always a client and a server. There is no difference between what a client can do and what a server can do.

    Connections are bi-directional.

    Client steps:
    \begin{enumerate}
        \item Find out the address of the machine you wish to connect to
        \item Make a socket (fd)
        \item \texttt{connect()} to the server
        \item Wrap socket descriptor for nicer IO (\texttt{dup()} before calling \texttt{fdopen()})
    \end{enumerate}

    Server steps:
    \begin{enumerate}
        \item Make a socket
        \item (Optional) set parameters
        \item \texttt{bind()} the socket to a port
        \item Set the socket to \texttt{listen()} for connections
        \item Call \texttt{accept()} to allow a connection (use the new fd to interact with the client)
    \end{enumerate}

    \texttt{ntohs} converts a 16 bit value from network representation to the machines normal ordering.

    Note that \texttt{accept()} is a blocking call, so fork or create a pthread or \textit{use a non-blocking call but don't actually tho}.

    \textbf{IP headers} have a packet length $2^{16}$ bytes incl. header, protocol and TTL (reduced each time the packet reaches an interface, packet is dropped if TTL reaches 0).

    \texttt{ping} sends a message to a device, sends a copy back and calculates the travel time.

    \textbf{IPv4 structure} has 32 bits, divided into network and host parts eg
    \begin{center}
        \texttt{130.102 | 72.9}

        \texttt{10000010.01100100 | 01001000.00001001}
    \end{center}
    Increasing the number of bits in the network parts means a ``smaller'' network. When sending a message, the network layer needs to make a decision:
    \begin{enumerate}
        \item Send direct to the destination (find the MAC of the destination)
        \item Send via another mahcine (find the MAC of the intermediary)
    \end{enumerate}
    An organisation's network can be divided into \textbf{subnets}. A host can directly communicate with everything on the same subnet. Broadcasts will reach all hosts in the subnet. Network $\leftrightarrow$ subnet. We can describe the subnet in two ways, CIDR notation or subnet mask.
    
    130.102.0.0/16 means set all host bits to 0 and the value after / is how many bits are in the network part.
    
    130.102.12.0/24 is subnet of all addresses starting with 130.102.12 for example. Note that the /x of CIDR notation does not need to fall on a byte boundary.
    
    Each subnet will have two addresses reserved: all host bits = zero (minimum host address) and all host bits = one (maximum host address). So subnet \texttt{A.B.C.D/x} has $32-x$ host bits and $2^{32-x} - 2$ usable host addresses (31 is a special case).

    A \textbf{netmask} is a bit pattern which will map any IP address to the corresponding network address.
    \begin{itemize}
        \item Set all network bits to 1
        \item Set all host bits to 0.
    \end{itemize}
    For example, \texttt{/24} with mask \texttt{255.255.255.0}.
    \begin{itemize}
        \item 130.102.24.17 $\to$ 130.102.24.0
        \item 130.102.24.250 $\to$ 130.102.24.0
        \item 130.102.21.16 $\to$ 130.102.21.0
    \end{itemize}
    Needs to be a contiguous string of 1s. Example:
    \texttt{130.102.160.0/20} is

    \texttt{\color{blue}130.102.1010\color{black}0000.00000000} network bits

    \texttt{\color{blue}255.255.1111\color{black}0000.00000000} netmask

    \texttt{255.255.240.0}
    \begin{itemize}
        \item 130.102.163.19 $\to$ 130.102.160.0 --- yes
        \item 130.102.171.99 $\to$ 130.102.160.0 --- yes
        \item 130.102.176.14 $\to$ 130.102.176.0 --- yes
    \end{itemize}

    What is the broadcast address for use by 117.98.141.19 netmask = 255.254.0.0?
    Netmask tells us that the network is 117.98.0.0/15 or

    01110101.01100010.00000000.00000000/15

    Setting the $32-15 = 17$ least significant bits to 1 gives

    01110101.0110001\color{blue}0.00000000.00000000\color{black}/15
    
    01110101.0110001\color{blue}1.11111111.11111111\color{black}/15 = 117.99.255.255

    Give the CIDR form and netmask for the largest network which includes 100.89.19.80, 100.89.19.82 and does not include 100.89.19.97
    \tiny
    \texttt{yes 100.89.19.80 01100100...00010011.01\color{blue}0\color{black}10000}
    
    \texttt{yes 100.89.19.82 01100100...00010011.01\color{blue}0\color{black}10010}
    
    \texttt{ no 100.89.19.97 01100100...00010011.01\color{blue}1\color{black}00001}

    \tiny
    So 100.89.18.80/26 is as big as possible without including 97. The netmask is 255.255.255.224

    \textbf{Special networks}
    \begin{itemize}
        \item 10.0.0.0/8
        \item 172.16.0.0/12
        \item 192.168.0.0/16
        \item 169.254.0.0/16 (for auto config)
    \end{itemize}
    All addresses in 127.0.0.0/8 are loopback addresses ($2^{24} - 2$ addresses)

    \textbf{NAT} Host $X$ wants to connect with address $G$ (sends \{src-ip=X, src-port=sp, dest-ip=G, dest-port=80\}). Packet arrives at $G$. $G$ tries to reply with \{src-ip=G, src-port=80, dest-ip=X, dest-port=sp\} but reply doesn't go anywhere because nobody knows where $X$ is.

    NAT is network address translation.
    \begin{enumerate}
        \item $X \to \ldots \to R \to \ldots \to G$
        
        \{src-ip=X, src-port=sp, dest-ip=G, dest-port=80\}
        \item Packet arrives at $R$
        \item $R$ modifies address information
        
        \{src-ip=R, src-port=np, dest-ip=G, dest-port=80\}
        \item ...
        \item $G$ receives packet and replies
        
        \{src-ip=G, src-port=80, dest-ip=R, dest-port=np\}
        \item ... 
        \item $R$ receives packet and modifies info
        
        \{src-ip=G, src-port=80, dest-ip=X, dest-port=sp\}
        \item $X$ receives the message
    \end{enumerate}

    NAT only works because $R$ remembers that port $np$ corresponds to port $sp$ on $X$. $R$ does not need to be directly connected to $X$ or $G$.

    \textbf{DNS} IP packets need to use IP addresses. Map names to IP addresses. Hosts file is still used but useless.

    DNS is domain name service. Each ``domain'' will have at least two servers which know the name to address mapping for that domain. Have a collection of root nameservers. The root servers know the information for the nameservers for the TLDs. Those servers each know the nameservers for subdomains. DNS is essentially a distributed phonebook.
 
    Queries are UDP messages. Servers can operate iteratively or recursively. DNS responses have TTL (more stable mappings will have longer TTL). Load balancing: different requests for the same name could get different answers. Give answers which are close to the query source. DNS domains are independent of networks.

    \texttt{HTTP} is HyperText transfer protocol. Runs on top of TCP (so layer 5). We send:
    \begin{itemize}
        \item \texttt{GET /} (i want a page, this is the name)
        \item \texttt{HTTP/1.1} (the protocol we are using)
        \item A newline
    \end{itemize}
    We receive:
    \begin{itemize}
        \item \texttt{HTTP/1.1 400 Bad Request} (response is in this protocol, status code, readable version of the status)
        \item Headers (Content-Type: text/html (what sort of file), Content-Length: 173 (how big is the file))
        \item Blank line
        \item HTML content telling us it was a bad request
    \end{itemize}

    Note ACK means acknowledgment, so go BACK!!!!

    \subsection*{C Standard Library}
    \begin{lstlisting}
#include <stdio.h>
int printf  (const char *format, ...);
int fprintf (FILE *stream, const char *format, ...);
int sprint  (char *str, const char *format, ...);
int snprintf(char *str, size_t size, const char  
            *format, ...);
(returns -1 on error)
int scanf(const char *format, ...);
int fscanf(FILE *stream, const char *format, ...);
int sscanf(const char *str, const char *format, ...);
(returns EOF or number of tokens matched)
FILE *fopen(const char *path, const char *mode);
int fclose(FILE *fp);
int fgetc(FILE *stream);
char *fgets(char *s, int size, FILE *stream);
int feof(FILE *stream);
(returns NULL, EOF, EOF and NULL on errors)
feof is only set after a failed read!
#include <stdlib.h>
void *malloc(size_t size);
void free(void *ptr);
void *calloc(size_t nmemb, size_t size);
void *realloc(void *ptr, size_t size);
(returns NULL on error)
int atoi(const char *nptr);
long int strtol(const char *nptr, char **endptr, 
                int base);
(sets errno on error)
#include <string.h>
void *memset(void *s, int c, size_t n);
void *memcpy(void *dest, const void *src, size_t n);
char *strcpy(char *dest, const char *src);
char *strncpy(char *dest, const char *src, size_t n);
int strcmp(const char *s1, const char *s2);
    \end{lstlisting}

    \subsection*{System Calls}
    \begin{lstlisting}
#include <unistd.h>
pid_t fork(void);
(returns 0 to child, child PID to parent, -1 on error)
int execl(const char *path, const char *arg, ...);
int execlp(const char *file, const char *arg, ...);
(does not return on success)
int pipe(int pipefd[2]);
#include <sys/wait.h>
pid_t wait(int *stat_loc);
pid_t waitpid(pid_t pid, int *stat_loc, int options);
(returns PID if reaped, 0 if nothing available)
    \end{lstlisting}

    \subsection*{Threading}
    \begin{lstlisting}
#include <pthread.h> (gcc -pthread)
int pthread_create(pthread_t *thread, const 
    pthread_attr_t *attr, 
    void *(*start_routine) (void *), void *arg);
int pthread_join(pthread_t thread, void **retval);
int pthread_mutex_init(pthread_mutex_t *restrict
    mutex, const pthread_mutexattr_t *restrict attr);
int pthread_mutex_destroy(pthread_mutex_t *mutex);
    \end{lstlisting}

    \subsection*{Networking}
    \begin{lstlisting}
#include <sys/types.h>
#include <sys/socket.h>
#include <netdb.h>
int socket(int domain, int type, int protocol);
int connect(int sockfd, const struct sockaddr *addr, 
            socklen_t addrlen); 
int bind(int sockfd, const struct sockaddr *addr,  
        socklen_t addrlen);
int listen(int sockfd, int backlog);
int accept(int sockfd, struct sockaddr *addr, 
        socklen_t *addrlen);

int getaddrinfo(const char *node, const char *service, 
                const struct addrinfo *hints, 
                struct addrinfo **res);
void freeaddrinfo(struct addrinfo *res);
    \end{lstlisting}

    \subsection*{Misc}
    \begin{lstlisting}
void qsort(void *base, size_t nmemb, size_t size,
            int (*compar)(const void *, const void *));
    \end{lstlisting}

    \subsection*{Read line example}
    \begin{lstlisting}
char* read_line(FILE* file) {
    char* result = malloc(sizeof(char)*40);
    int position = 0;
    int next = 0;

    while (1) {
        next = fgetc(file);
        if (next == EOF || next == '\n') {
            result[position] = '\0';
            result = reverse(result);
            return result;
        } else {
            result[position++] = (char)next;
        }
        if (position > 30) {
            result = realloc(result, sizeof(char)*position+1);
        }
    }
}
    \end{lstlisting}

    \subsection*{Equations}
    \begin{align*}
        \text{Block Number} &= \left\lfloor \frac{\text{Address}}{\text{Block Size}}\right\rfloor \\
        \text{Offset} &= \text{Address } \% \text{ Block Size} \\
        \text{No. of Block Pointers} &= \frac{\text{Block Size}}{\text{Block Pointer Size}} \\
        \text{Subdirectories} &= \text{Link Count} -2
    \end{align*}

    \subsection*{2018 Question 11}
    \lstinputlisting{code/2018-11.c}

\end{multicols}

\end{document}
